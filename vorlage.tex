\documentclass{article}

%links im inhaltsverzeichnis verwenden
\usepackage{hyperref}

%Damit die Überschrift vom Inhaltsverzeichnis auf Deutsch angezeigt wird
\usepackage[ngerman]{babel}

% Damit umlaute richtig angezeigt werden
\usepackage[utf8]{inputenc}

% Randabstände anpassen
\usepackage[left=2cm,right=2cm]{geometry}

% wenn man blocksatz verwenden will. Nutze \justify  oder \section{justify} für blocksatz
\usepackage{ragged2e}

% schriftart arial verwenden
\usepackage{arial}

\hypersetup{
    colorlinks=true,
    linkcolor=blue,
    filecolor=magenta,      
    urlcolor=cyan,
}

\begin{document}

%Inhaltsverzeichnis
\newpage
\tableofcontents
\newpage

%Beginn des Dokuments

\section{Einleitung}
Dieser Abschnitt führt in das Projekt ein.

\subsection{Hintergrund}
Dieser Abschnitt liefert Hintergrundinformationen zum Projekt.

\subsubsection{Verwandte Arbeit}
Dieser Abschnitt diskutiert verwandte Arbeit in diesem Bereich.

\section{Methodik}
Dieser Abschnitt beschreibt die Methodik, die im Projekt verwendet wird.

\subsection{Datensammlung}
Dieser Abschnitt beschreibt den Prozess der Datensammlung.

\subsection{Datenanalyse}
Dieser Abschnitt beschreibt den Prozess der Datenanalyse.

\section{Ergebnisse}
Dieser Abschnitt präsentiert die Ergebnisse des Projekts.

\subsection{Quantitative Ergebnisse}
Dieser Abschnitt präsentiert die quantitativen Ergebnisse.

\subsection{Qualitative Ergebnisse}
Dieser Abschnitt präsentiert die qualitativen Ergebnisse.

\section{Schlussfolgerung}
Dieser Abschnitt fasst das Projekt und seine Ergebnisse zusammen.



\end{document}


